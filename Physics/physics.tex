\documentclass{article}
\usepackage{graphicx} % Required for inserting images
\usepackage{amsmath}

\renewcommand{\today}{\ifcase \month \or January\or February\or March\or%
April\or May \or June\or July\or August\or September\or October\or November\or %
December\fi, \number \year} 

% \makeatletter
% \renewcommand{\@maketitle}{
%    \raggedright\begin{flushleft}
%       \Large\bfseries\@title\par
%       \medskip
%       \normalsize \@author\par
%       \medskip % Adds some space between author and date
%       \normalsize \@date\par
%    \end{flushleft}
% }
% \makeatother


\title{Physics\\A Reference Manual}
\author{Khalil Gatto}
\date{\today}

\begin{document}
\maketitle


\section*{Fundamentals of Electricity and Magnetism}

\subsection*{Coulomb's Law}
\textbf{Equation:} 
\[ F = k_e \frac{{|q_1 \cdot q_2|}}{{r^2}} \]
\textbf{Explanation:} 
Coulomb's Law calculates the electric force \( F \) between two charges, \( q_1 \) and \( q_2 \), separated by a distance \( r \). \( k_e \) is Coulomb's constant. The force is attractive if the charges are opposite and repulsive if they are the same.



\subsection*{Electric Field}
\textbf{Equation:} 
\[ E = \frac{F}{q} \]
\textbf{Explanation:} 
The electric field \( E \) at a point in space is defined as the electric force \( F \) experienced by a small positive test charge \( q \) placed at that point, divided by the magnitude of the charge.


\subsection*{Ohm's Law}
\textbf{Equation:} 
\[ V = IR \]
\textbf{Explanation:} 
Ohm's Law relates the voltage across a conductor \( V \), the current flowing through it \( I \), and its resistance \( R \). It states that the current is directly proportional to the voltage and inversely proportional to the resistance.

\subsection*{Gauss' Law and Symmetry}
Gauss' Law is a fundamental principle that relates the distribution of electric charge to the resulting electric field. The law is expressed as:
\begin{equation}
    \oint \vec{E} \cdot d\vec{A} = \frac{q_{\text{enc}}}{\varepsilon_0}
\end{equation}
where the total electric flux $\Phi_E$ through a closed surface is equal to the charge $q_{\text{enc}}$ enclosed by the surface divided by the vacuum permittivity $\varepsilon_0$.

\subsection*{Example: Spherical Symmetry}
For a point charge, the electric field is radially outward and the Gaussian surface is a sphere:
\begin{equation}
    E(4\pi r^2) = \frac{q}{\varepsilon_0} \quad \implies \quad E = \frac{q}{4\pi \varepsilon_0 r^2}
\end{equation}
where $r$ is the radius of the Gaussian sphere and $q$ is the enclosed charge.

\subsection*{Charges on Conductors}
For a conductor in electrostatic equilibrium, the charge resides on the surface, the electric field inside is zero, and on the surface, it's perpendicular to the surface.

\subsection*{Induced Charges on Conductors}
When a conductor is placed in an external electric field, charges within the conductor redistribute until the field inside is zero, creating induced charges on the surface.

\subsection*{Induced Charges on Conductors}
When a conductor is placed in an external electric field, charges within the conductor redistribute until the field inside is zero, creating induced charges on the surface.

\subsection*{Example: Solid Infinite Cylindrical Conductor}
For a long cylindrical conductor with charge per unit length $\lambda$, Gauss' Law can be applied with a cylindrical Gaussian surface:
\begin{equation}
    E(2\pi rL) = \frac{\lambda L}{\varepsilon_0} \quad \implies \quad E = \frac{\lambda}{2\pi\varepsilon_0 r}
\end{equation}
where $L$ is the length of the cylinder and $r$ is the radial distance from the axis.

\subsection*{Infinite Sheet of Charge}
For an infinite sheet of charge with surface charge density $\sigma$, the electric field is:
\begin{equation}
    E = \frac{\sigma}{2\varepsilon_0}
\end{equation}
directed perpendicularly away from the sheet.

\subsection*{Superposition}
The principle of superposition states that the total electric field created by multiple charges is the vector sum of the electric fields created by each charge individually:
\begin{equation}
    \vec{E}_{\text{total}} = \sum_{i=1}^{n} \vec{E}_i
\end{equation}
where $\vec{E}_i$ is the electric field due to the $i$th charge.


\subsection*{Faraday's Law of Electromagnetic Induction}
\textbf{Equation:} 
\[ \mathcal{E} = -\frac{d\Phi_B}{dt} \]
\textbf{Explanation:} 
Faraday's Law states that the induced electromotive force (EMF) \( \mathcal{E} \) in any closed circuit is equal to the negative rate of change of the magnetic flux \( \Phi_B \) through the circuit.

\subsection*{Ampere's Law}
\textbf{Equation:} 
\[ \oint \vec{B} \cdot d\vec{l} = \mu_0 I_{\text{enc}} \]
\textbf{Explanation:} 
Ampere's Law relates the integrated magnetic field \( \vec{B} \) around a closed loop to the electric current \( I_{\text{enc}} \) passing through any surface bounded by the loop. \( \mu_0 \) is the permeability of free space.

\subsection*{Maxwell's Equations}
\textbf{Equation:} 
\begin{itemize}
    \item Gauss's Law for Electricity: \( \oint \vec{E} \cdot d\vec{A} = \frac{Q_{\text{enc}}}{\varepsilon_0} \)
    \item Gauss's Law for Magnetism: \( \oint \vec{B} \cdot d\vec{A} = 0 \)
    \item Faraday's Law: \( \oint \vec{E} \cdot d\vec{l} = -\frac{d\Phi_B}{dt} \)
    \item Ampere's Law with Maxwell's addition: \( \oint \vec{B} \cdot d\vec{l} = \mu_0 (I_{\text{enc}} + \varepsilon_0 \frac{d\Phi_E}{dt}) \)
\end{itemize}
\textbf{Explanation:} 
Maxwell's Equations are a set of four equations that form the foundation of classical electromagnetism. They describe how electric and magnetic fields are generated by charges, currents, and changes of the fields.

\subsection*{Magnetic Force on a Moving Charge}
\textbf{Equation:} 
\[ \vec{F} = q\vec{v} \times \vec{B} \]
\textbf{Explanation:} 
A charge \( q \) moving with velocity \( \vec{v} \) in a magnetic field \( \vec{B} \) experiences a force \( \vec{F} \). This force is perpendicular to both the velocity and the magnetic field.

\subsection*{Magnetic Field Due to a Long Straight Wire}
\textbf{Equation:} 
\[ B = \frac{\mu_0 I}{2\pi r} \]
\textbf{Explanation:} 
The magnetic field \( B \) at a distance \( r \) from a long straight wire carrying current \( I \) is given by this formula, where \( \mu_0 \) is the permeability of free space.

\subsection*{Biot-Savart Law}
\textbf{Equation:} 
\[ d\vec{B} = \frac{\mu_0}{4\pi} \frac{Id\vec{s} \times \hat{r}}{r^2} \]
\textbf{Explanation:} 
The Biot-Savart Law calculates the differential magnetic field \( d\vec{B} \) produced at point P by a small segment of current-carrying wire \( Id\vec{s} \), where \( \hat{r} \) is the unit vector from the wire to point P.
\subsection*{Magnetic Field Produced by an Infinite Straight Wire}
\textbf{Equation:} 
\[ B = \frac{\mu_0 I}{2\pi r} \]
\textbf{Explanation:} 
For an infinitely long straight wire carrying a steady current \(I\), the magnetic field \(B\) at a distance \(r\) from the wire is given by this equation, where \(\mu_0\) is the permeability of free space. The direction of the field is given by the right-hand rule.

\subsection*{Force between Two Parallel Current Carrying Wires}
\textbf{Equation:} 
\[ F = \frac{\mu_0 I_1 I_2 L}{2\pi d} \]
\textbf{Explanation:} 
Two parallel wires carrying currents \(I_1\) and \(I_2\) over a length \(L\), separated by a distance \(d\), exert a force \(F\) on each other. The force is attractive if the currents are in the same direction and repulsive if they are in opposite directions.

\subsection*{Magnetic Field along the Axis of a Current Loop}
\textbf{Equation:} 
\[ B = \frac{\mu_0 I R^2}{2(R^2 + z^2)^{3/2}} \]
\textbf{Explanation:} 
The magnetic field \(B\) at a point along the axis of a circular loop of radius \(R\) carrying current \(I\) at a distance \(z\) from the center of the loop is given by this equation. The field is strongest at the center of the loop and decreases with distance from the loop.

\subsection*{The Off-Axis Magnetic Field of a Current Carrying Loop}
\textbf{Equation:} 
Complex - involves elliptic integrals.
\textbf{Explanation:} 
The off-axis magnetic field due to a current carrying loop is more complex and typically requires numerical methods or approximations for practical calculation. It involves integrating the contributions of each element of the loop over its circumference considering the position relative









\subsection*{Magnetic Flux}
\textbf{Equation:} 
\[ \Phi_B = \int \vec{B} \cdot d\vec{A} \]
\textbf{Explanation:} 
Magnetic flux \( \Phi_B \) through a surface is the integral of the magnetic field \( \vec{B} \) over that surface \( d\vec{A} \). It represents the number of magnetic field lines passing through the surface.

\subsection*{Ampere's Circuital Law}
\textbf{Equation:} 
\[ \oint \vec{B} \cdot d\vec{l} = \mu_0 I_{\text{enc}} \]
\textbf{Explanation:} 
Ampere's Circuital Law relates the integrated magnetic field \( \vec{B} \) around a closed loop to the total electric current \( I_{\text{enc}} \) passing through the loop.

\subsection*{Lorentz Force Law}
\textbf{Equation:} 
\[ \vec{F} = q(\vec{E} + \vec{v} \times \vec{B}) \]
\textbf{Explanation:} 
The Lorentz force law gives the total force \( \vec{F} \) experienced by a charge \( q \) moving with velocity \( \vec{v} \) in the presence of electric \( \vec{E} \) and magnetic \( \vec{B} \) fields.

\subsection*{Magnetic Moment}
\textbf{Equation:} 
\[ \vec{\mu} = I\vec{A} \]
\textbf{Explanation:} 
The magnetic moment \( \vec{\mu} \) of a current loop is the product of the current \( I \) and the area vector \( \vec{A} \) of the loop. It determines the torque the loop experiences in a magnetic field.

\subsection*{Magnetic Field of a Solenoid}
\textbf{Equation:} 
\[ B = \mu_0 n I \]
\textbf{Explanation:} 
The magnetic field \( B \) inside a long solenoid with \( n \) turns per unit length carrying current \( I \) is uniform and given by this equation.

\subsection*{Force between Two Parallel Currents}
\textbf{Equation:} 
\[ F = \frac{\mu_0 I_1 I_2 L}{2\pi d} \]
\textbf{Explanation:} 
This formula calculates the force per unit length \( F/L \) between two long parallel wires carrying currents \( I_1 \) and \( I_2 \) separated by distance \( d \). The force is attractive for currents in the same direction and repulsive for opposite directions.

\subsection*{Magnetic Materials and Susceptibility}
\textbf{Equation:} 
\[ \chi_m = \frac{M}{H} \]
\textbf{Explanation:} 
The magnetic susceptibility \( \chi_m \) of a material is the ratio of its magnetization \( M \) to the applied magnetic field \( H \). It indicates how easily the material can be magnetized.
A material with a magnetization of \( 4 \, \text{A/m} \) in a \( 2 \, \text{T/m} \) field has a susceptibility of 2.

\subsection*{Motional emf}
\textbf{Equation:} 
\[ \mathcal{E} = B \ell v \sin(\theta) \]
\textbf{Explanation:} 
Motional emf (\( \mathcal{E} \)) is the electromotive force induced in a conductor moving through a magnetic field. This occurs due to the Lorentz force acting on the charges within the conductor. Here, \( B \) is the magnetic field strength, \( \ell \) is the length of the conductor within the magnetic field, \( v \) is the velocity of the conductor relative to the magnetic field, and \( \theta \) is the angle between the velocity and the magnetic field. The sin(\( \theta \)) component determines the effective component of velocity that is perpendicular to the magnetic field.

\section*{Forces and Torques on Currents in Electricity and Magnetism}

\subsection*{Force on a Straight Current Segment}
\textbf{Equation:} 
\[ \vec{F} = I\vec{L} \times \vec{B} \]
\textbf{Explanation:} 
A straight segment of wire carrying a current \( I \) in a magnetic field \( \vec{B} \) experiences a force \( \vec{F} \). The length of the wire in the field is represented by \( \vec{L} \), and the force is perpendicular to both \( \vec{L} \) and \( \vec{B} \).

\subsection*{Force on a Curved Current Segment}
\textbf{Equation:} 
\[ d\vec{F} = I d\vec{l} \times \vec{B} \]
\textbf{Explanation:} 
For a curved segment of current-carrying wire, the differential force \( d\vec{F} \) on a small segment \( d\vec{l} \) is given by this equation. The total force is the integral of \( d\vec{F} \) over the length of the wire.

\subsection*{Force on a Current Loop}
\textbf{Equation:} 
\[ \vec{F} = \oint (I d\vec{l} \times \vec{B}) \]
\textbf{Explanation:} 
The net force on a current loop in a uniform magnetic field is often zero because the forces on opposite segments of the loop cancel out. However, in a non-uniform field, the loop can experience a net force.

\subsection*{Torque on a Current Loop}
\textbf{Equation:} 
\[ \vec{\tau} = \vec{\mu} \times \vec{B} \]
\textbf{Explanation:} 
A current loop in a magnetic field experiences a torque \( \vec{\tau} \), which tends to rotate the loop. Here, \( \vec{\mu} = I\vec{A} \) is the magnetic dipole moment of the loop with area \( \vec{A} \), and \( \vec{B} \) is the magnetic field.

\subsection*{Dipole Moment of Current Loop}
\textbf{Equation:} 
\[ \vec{\mu} = I\vec{A} \]
\textbf{Explanation:} 
The magnetic dipole moment \( \vec{\mu} \) of a current loop is the product of the current \( I \) and the area vector \( \vec{A} \) of the loop. It represents the strength and orientation of the loop's magnetic effect.

\subsection*{Potential Energy of Dipole in Magnetic Field}
\textbf{Equation:} 
\[ U = -\vec{\mu} \cdot \vec{B} \]
\textbf{Explanation:} 
The potential energy \( U \) of a magnetic dipole \( \vec{\mu} \) in a magnetic field \( \vec{B} \) is given by this equation. It represents the work done to rotate the dipole from its stable equilibrium position to the current orientation.


\end{document}