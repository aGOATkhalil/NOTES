\documentclass{article}
\usepackage{graphicx} % Required for inserting images
\usepackage{amsmath}

\renewcommand{\today}{\ifcase \month \or January\or February\or March\or%
April\or May \or June\or July\or August\or September\or October\or November\or %
December\fi, \number \year} 

% \makeatletter
% \renewcommand{\@maketitle}{
%    \raggedright\begin{flushleft}
%       \Large\bfseries\@title\par
%       \medskip
%       \normalsize \@author\par
%       \medskip % Adds some space between author and date
%       \normalsize \@date\par
%    \end{flushleft}
% }
% \makeatother


\title{Physics\\A Reference Manual}
\author{Khalil Gatto}
\date{\today}

\begin{document}
\maketitle


\section*{Fundamentals of Electricity and Magnetism}

\subsection*{1. Coulomb's Law}
\textbf{Equation:} 
\[ F = k_e \frac{{|q_1 \cdot q_2|}}{{r^2}} \]
\textbf{Explanation:} 
Coulomb's Law calculates the electric force \( F \) between two charges, \( q_1 \) and \( q_2 \), separated by a distance \( r \). \( k_e \) is Coulomb's constant. The force is attractive if the charges are opposite and repulsive if they are the same.
\textbf{Example:}
If two charges, +1 C and -1 C, are 1 meter apart, the force between them is calculated using \( k_e = 8.987 \times 10^9 \, \text{N m}^2/\text{C}^2 \).

\subsection*{2. Electric Field}
\textbf{Equation:} 
\[ E = \frac{F}{q} \]
\textbf{Explanation:} 
The electric field \( E \) at a point in space is defined as the electric force \( F \) experienced by a small positive test charge \( q \) placed at that point, divided by the magnitude of the charge.
\textbf{Example:}
If a 1 C charge experiences a force of 10 N, the electric field at that point is 10 N/C.

\subsection*{3. Ohm's Law}
\textbf{Equation:} 
\[ V = IR \]
\textbf{Explanation:} 
Ohm's Law relates the voltage across a conductor \( V \), the current flowing through it \( I \), and its resistance \( R \). It states that the current is directly proportional to the voltage and inversely proportional to the resistance.
\textbf{Example:}
If a 10 V battery is connected across a resistor of 5 ohms, the current flowing through the circuit is \( I = \frac{V}{R} = 2 \) A.

\subsection*{4. Faraday's Law of Electromagnetic Induction}
\textbf{Equation:} 
\[ \mathcal{E} = -\frac{d\Phi_B}{dt} \]
\textbf{Explanation:} 
Faraday's Law states that the induced electromotive force (EMF) \( \mathcal{E} \) in any closed circuit is equal to the negative rate of change of the magnetic flux \( \Phi_B \) through the circuit.
\textbf{Example:}
If the magnetic flux through a loop changes by 0.02 Wb in 0.01 seconds, the induced EMF is \( \mathcal{E} = -2 \) V.

\subsection*{5. Ampere's Law}
\textbf{Equation:} 
\[ \oint \vec{B} \cdot d\vec{l} = \mu_0 I_{\text{enc}} \]
\textbf{Explanation:} 
Ampere's Law relates the integrated magnetic field \( \vec{B} \) around a closed loop to the electric current \( I_{\text{enc}} \) passing through any surface bounded by the loop. \( \mu_0 \) is the permeability of free space.
\textbf{Example:}
For a long straight wire carrying a current of 5 A, the magnetic field at a distance of 2 meters from the wire can be calculated using Ampere's Law.

\subsection*{6. Maxwell's Equations}
\textbf{Equation:} 
\begin{itemize}
    \item Gauss's Law for Electricity: \( \oint \vec{E} \cdot d\vec{A} = \frac{Q_{\text{enc}}}{\varepsilon_0} \)
    \item Gauss's Law for Magnetism: \( \oint \vec{B} \cdot d\vec{A} = 0 \)
    \item Faraday's Law: \( \oint \vec{E} \cdot d\vec{l} = -\frac{d\Phi_B}{dt} \)
    \item Ampere's Law with Maxwell's addition: \( \oint \vec{B} \cdot d\vec{l} = \mu_0 (I_{\text{enc}} + \varepsilon_0 \frac{d\Phi_E}{dt}) \)
\end{itemize}
\textbf{Explanation:} 
Maxwell's Equations are a set of four equations that form the foundation of classical electromagnetism. They describe how electric and magnetic fields are generated by charges, currents, and changes of the fields.
\textbf{Example:} 
Using Gauss's Law for Electricity, the electric field outside a uniformly charged sphere can be found by considering a Gaussian surface outside the sphere.

\subsection*{1. Magnetic Force on a Moving Charge}
\textbf{Equation:} 
\[ \vec{F} = q\vec{v} \times \vec{B} \]
\textbf{Explanation:} 
A charge \( q \) moving with velocity \( \vec{v} \) in a magnetic field \( \vec{B} \) experiences a force \( \vec{F} \). This force is perpendicular to both the velocity and the magnetic field.
\textbf{Example:}
A charge of +1 C moving at \( 1 \, \text{m/s} \) perpendicular to a magnetic field of \( 1 \, \text{T} \) experiences a force of 1 N.

\subsection*{2. Magnetic Field Due to a Long Straight Wire}
\textbf{Equation:} 
\[ B = \frac{\mu_0 I}{2\pi r} \]
\textbf{Explanation:} 
The magnetic field \( B \) at a distance \( r \) from a long straight wire carrying current \( I \) is given by this formula, where \( \mu_0 \) is the permeability of free space.
\textbf{Example:}
A wire carrying a current of 2 A will produce a magnetic field of \( 4 \times 10^{-7} \, \text{T} \) at a distance of 1 m from the wire.

\subsection*{3. Biot-Savart Law}
\textbf{Equation:} 
\[ d\vec{B} = \frac{\mu_0}{4\pi} \frac{Id\vec{s} \times \hat{r}}{r^2} \]
\textbf{Explanation:} 
The Biot-Savart Law calculates the differential magnetic field \( d\vec{B} \) produced at point P by a small segment of current-carrying wire \( Id\vec{s} \), where \( \hat{r} \) is the unit vector from the wire to point P.
\textbf{Example:}
Calculating the magnetic field at a point away from a small segment of current-carrying wire requires integrating this expression over the length of the wire.

\subsection*{4. Magnetic Flux}
\textbf{Equation:} 
\[ \Phi_B = \int \vec{B} \cdot d\vec{A} \]
\textbf{Explanation:} 
Magnetic flux \( \Phi_B \) through a surface is the integral of the magnetic field \( \vec{B} \) over that surface \( d\vec{A} \). It represents the number of magnetic field lines passing through the surface.
\textbf{Example:}
If a magnetic field of 2 T passes uniformly through a 1 \( \text{m}^2 \) area, the magnetic flux is 2 Weber.

\subsection*{5. Ampere's Circuital Law}
\textbf{Equation:} 
\[ \oint \vec{B} \cdot d\vec{l} = \mu_0 I_{\text{enc}} \]
\textbf{Explanation:} 
Ampere's Circuital Law relates the integrated magnetic field \( \vec{B} \) around a closed loop to the total electric current \( I_{\text{enc}} \) passing through the loop.
\textbf{Example:}
For a long, straight wire carrying a current of 5 A, the magnetic field can be calculated at any point in a circular path around the wire.

\subsection*{6. Lorentz Force Law}
\textbf{Equation:} 
\[ \vec{F} = q(\vec{E} + \vec{v} \times \vec{B}) \]
\textbf{Explanation:} 
The Lorentz force law gives the total force \( \vec{F} \) experienced by a charge \( q \) moving with velocity \( \vec{v} \) in the presence of electric \( \vec{E} \) and magnetic \( \vec{B} \) fields.
\textbf{Example:}
A +1 C charge moving with a velocity of \( 1 \, \text{m/s} \) in a region with both a 1 T magnetic field and a 1 V/m electric field will experience a force.

\subsection*{7. Magnetic Moment}
\textbf{Equation:} 
\[ \vec{\mu} = I\vec{A} \]
\textbf{Explanation:} 
The magnetic moment \( \vec{\mu} \) of a current loop is the product of the current \( I \) and the area vector \( \vec{A} \) of the loop. It determines the torque the loop experiences in a magnetic field.
\textbf{Example:}
A loop carrying 2 A of current with an area of 1 \( \text{m}^2 \) has a magnetic moment of 2 \( \text{A m}^2 \).

\subsection*{8. Magnetic Field of a Solenoid}
\textbf{Equation:} 
\[ B = \mu_0 n I \]
\textbf{Explanation:} 
The magnetic field \( B \) inside a long solenoid with \( n \) turns per unit length carrying current \( I \) is uniform and given by this equation.
\textbf{Example:}
A solenoid with 1000 turns per meter carrying 0.5 A will have a magnetic field of \( 6.28 \times 10^{-4} \, \text{T} \) inside it.

\subsection*{9. Force between Two Parallel Currents}
\textbf{Equation:} 
\[ F = \frac{\mu_0 I_1 I_2 L}{2\pi d} \]
\textbf{Explanation:} 
This formula calculates the force per unit length \( F/L \) between two long parallel wires carrying currents \( I_1 \) and \( I_2 \) separated by distance \( d \). The force is attractive for currents in the same direction and repulsive for opposite directions.
\textbf{Example:}
Two parallel wires 1 m apart carrying 2 A and 3 A respectively experience a force per meter of \( 4 \times 10^{-7} \, \text{N/m} \).

\subsection*{10. Magnetic Materials and Susceptibility}
\textbf{Equation:} 
\[ \chi_m = \frac{M}{H} \]
\textbf{Explanation:} 
The magnetic susceptibility \( \chi_m \) of a material is the ratio of its magnetization \( M \) to the applied magnetic field \( H \). It indicates how easily the material can be magnetized.
\textbf{Example:}
A material with a magnetization of \( 4 \, \text{A/m} \) in a \( 2 \, \text{T/m} \) field has a susceptibility of 2.

\subsection*{Motional emf}
\textbf{Equation:} 
\[ \mathcal{E} = B \ell v \sin(\theta) \]
\textbf{Explanation:} 
Motional emf (\( \mathcal{E} \)) is the electromotive force induced in a conductor moving through a magnetic field. This occurs due to the Lorentz force acting on the charges within the conductor. Here, \( B \) is the magnetic field strength, \( \ell \) is the length of the conductor within the magnetic field, \( v \) is the velocity of the conductor relative to the magnetic field, and \( \theta \) is the angle between the velocity and the magnetic field. The sin(\( \theta \)) component determines the effective component of velocity that is perpendicular to the magnetic field.

\textbf{Example:}
Consider a rod of length 1 m moving at a speed of 2 m/s perpendicular (\( \theta = 90^\circ \)) to a magnetic field of strength 0.5 T. The motional emf induced in the rod is calculated as \( \mathcal{E} = 0.5 \times 1 \times 2 \times \sin(90^\circ) = 1 \) V.

\section*{Forces and Torques on Currents in Electricity and Magnetism}

\subsection*{1) Force on a Straight Current Segment}
\textbf{Equation:} 
\[ \vec{F} = I\vec{L} \times \vec{B} \]
\textbf{Explanation:} 
A straight segment of wire carrying a current \( I \) in a magnetic field \( \vec{B} \) experiences a force \( \vec{F} \). The length of the wire in the field is represented by \( \vec{L} \), and the force is perpendicular to both \( \vec{L} \) and \( \vec{B} \).

\subsection*{2) Force on a Curved Current Segment}
\textbf{Equation:} 
\[ d\vec{F} = I d\vec{l} \times \vec{B} \]
\textbf{Explanation:} 
For a curved segment of current-carrying wire, the differential force \( d\vec{F} \) on a small segment \( d\vec{l} \) is given by this equation. The total force is the integral of \( d\vec{F} \) over the length of the wire.

\subsection*{3) Force on a Current Loop}
\textbf{Equation:} 
\[ \vec{F} = \oint (I d\vec{l} \times \vec{B}) \]
\textbf{Explanation:} 
The net force on a current loop in a uniform magnetic field is often zero because the forces on opposite segments of the loop cancel out. However, in a non-uniform field, the loop can experience a net force.

\subsection*{4) Torque on a Current Loop}
\textbf{Equation:} 
\[ \vec{\tau} = \vec{\mu} \times \vec{B} \]
\textbf{Explanation:} 
A current loop in a magnetic field experiences a torque \( \vec{\tau} \), which tends to rotate the loop. Here, \( \vec{\mu} = I\vec{A} \) is the magnetic dipole moment of the loop with area \( \vec{A} \), and \( \vec{B} \) is the magnetic field.

\subsection*{5) Dipole Moment of Current Loop}
\textbf{Equation:} 
\[ \vec{\mu} = I\vec{A} \]
\textbf{Explanation:} 
The magnetic dipole moment \( \vec{\mu} \) of a current loop is the product of the current \( I \) and the area vector \( \vec{A} \) of the loop. It represents the strength and orientation of the loop's magnetic effect.

\subsection*{6) Potential Energy of Dipole in Magnetic Field}
\textbf{Equation:} 
\[ U = -\vec{\mu} \cdot \vec{B} \]
\textbf{Explanation:} 
The potential energy \( U \) of a magnetic dipole \( \vec{\mu} \) in a magnetic field \( \vec{B} \) is given by this equation. It represents the work done to rotate the dipole from its stable equilibrium position to the current orientation.


\end{document}