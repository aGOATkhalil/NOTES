\documentclass{article}
\usepackage{graphicx} % Required for inserting images
\usepackage{amsmath} % for math environments and more
\usepackage{amssymb} % for extended symbol collection
\usepackage{setspace}
\usepackage{tikz}
\usetikzlibrary{angles,quotes} % for the angle labeling
\graphicspath{{C:/Users/Khalil Gatto/Desktop/Pictures/}}


\renewcommand{\today}{\ifcase \month \or January\or February\or March\or%
April\or May \or June\or July\or August\or September\or October\or November\or %
December\fi, \number \year} 


\title{Mathematics\\A Reference Manual}
\author{Written by Khalil Gatto}
\date{\today}

\begin{document}
\maketitle

\begin{abstract}
    This document covers Calculus, Linear Algebra, Math Analysis and Differential Equations.
\end{abstract}

\section*{Calculus 3}
This section largely follows Chapters 12-16 in Calculus Early Transcendentals.

\subsection*{Chapter 12: Vectors and Geometry of Space}

\paragraph{Distance Formula in 3D}
\begin{equation} 
    \label{eq:Distance Formula in 3D}
    |P_1P_2| = \sqrt{(x_2 - x_1)^2 + (y_2 - y_1)^2 + (z_2 - z_1)^2} 
\end{equation}

\paragraph{Equation of a Sphere}\mbox{}\\
r: Radius of phere
\newline
C(h,k,l): Center of sphere
\begin{equation}
    \label{eq:Equation of a Sphere}
    (x - h)^2 + (y-k)^2 + (z-l)^2 = r^2
\end{equation}

\paragraph{Unit Vectors}\mbox{}\\ %the mbox command is a cheat to force a newline in paragraph
Suppose that \(\vec{a} \neq 0\) and that \(\vec{u}\) has the same direction of \(\vec{a}\) :
\begin{equation}
    \label{eq:Unit Vector}
    u = \frac{1}{|a|} a = \frac{a}{|a|}
\end{equation}

\paragraph{Dot Products}\mbox{}\\
\begin{equation}
    \label{eq:Dot Product}
    a \cdot b = a_1b_1 + a_2b_2 +...a_nb_n
\end{equation}
Properties include:\newline
1. Output is a scalar.\hspace{1cm}2. \(a \cdot a = |a|^2\)\hspace{1cm}
3. \(a \cdot b = b \cdot a\)\hspace{1cm}\newline
4. \(a \cdot (b+c) = a \cdot b + a \cdot c\)\hspace{1cm}5. \(ca \cdot b = c(b \cdot a) = a \cdot (cb)\)
\newline\newline
Theorem: If \(\theta\) is the angle between two vectors, then\newline
\begin{equation}
    \label{eq: Theta between two vectors}
    a \cdot b = |a||b|\cos{\theta}
\end{equation}
\begin{equation}
    \label{eq: Theta dot product 2}
    \cos{\theta} = \frac{a \cdot b}{|a||b|} 
\end{equation}
\newline
Two vectors are orthogonal if \(a \cdot b = 0\)\newline

\paragraph{Projections}\mbox{}\\
Scalar projection of \(\vec{b}\) onto \(\vec{a}\)... aka component of b along a: 
\begin{equation}
    \label{eq: scalar projection}
    \text{comp}_{\vec{a}}{\vec{b}} = \frac{a \cdot b}{|a|}
\end{equation}
\newline
Vector projection of \(\vec{b}\) onto \(\vec{a}\): 
\begin{equation}
    \label{eq: vector projection}
    \text{proj}_{\vec{a}}{\vec{b}} = \left(\frac{a \cdot b}{|a|}\right) \frac{a}{|a|} 
    = \left(\frac{a \cdot b}{|a|^2}\right) a
\end{equation}

\paragraph{Associated Inequalities and Identities}\mbox{}\\
\begin{doublespace}
Cauchy-Schwartz Inequality:\hspace{1cm}\(|a \cdot b| \leq |a||b|\)\\
Triangle Inequality: \hspace{1cm}\((a + b) \leq |a| + |b|\)\\
Parallelogram Identity: \hspace{1cm} \(|a+b|^2 + |a-b|^2 = 2|a|^2 +  2|b|^2\)
\end{doublespace}

\paragraph{Cross Product}\mbox{}\\
Suppose \(a = \langle a_1, a_2, a_3 \rangle\) and \(b = \langle b_1, b_2, b_3 \rangle\) 
or \( \vec{a} = \begin{bmatrix} a_1 \\ a_2 \\ a_3 \end{bmatrix} \) and \( \vec{b} = \begin{bmatrix} b_1 \\ b_2 \\ b_3 \end{bmatrix} \)
then\\
\begin{equation}
    \label{corss product}
    a \times b = \langle a_2b_3 - a_3b_2, a_3b_1 - a_1b_3, a_1b_2 - a_2b_1 \rangle
\end{equation}\\
Which really comes from the following. Instead do this:\\
\[ 
\vec{a} \times \vec{b} = 
\begin{vmatrix}
\mathbf{i} & \mathbf{j} & \mathbf{k} \\
a_1 & a_2 & a_3 \\
b_1 & b_2 & b_3
\end{vmatrix} 
\]
Expanding this determinant, we get:
\[ 
\vec{a} \times \vec{b} = \mathbf{i}
\begin{vmatrix}
a_2 & a_3 \\
b_2 & b_3 
\end{vmatrix}
- \mathbf{j}
\begin{vmatrix}
a_1 & a_3 \\
b_1 & b_3
\end{vmatrix}
+ \mathbf{k}
\begin{vmatrix}
a_1 & a_2 \\
b_1 & b_2
\end{vmatrix}
\]
\\\\\\
Properties:
\begin{enumerate}
    \item The resultant vector is orthogonal to both a and b.
    \item \(a \times b = |a||b|\sin{\theta}\)
    \item vectors a and b are parallel if \(a \times b = 0\)
    \item \( \mathbf{a} \times \mathbf{b} = -\mathbf{b} \times \mathbf{a} \)
    \item \( (c\mathbf{a}) \times \mathbf{b} = c(\mathbf{a} \times \mathbf{b}) = \mathbf{a} \times (c\mathbf{b}) \)
    \item \( \mathbf{a} \times (\mathbf{b} + \mathbf{c}) = \mathbf{a} \times \mathbf{b} + \mathbf{a} \times \mathbf{c} \)
    \item \( (\mathbf{a} + \mathbf{b}) \times \mathbf{c} = \mathbf{a} \times \mathbf{c} + \mathbf{b} \times \mathbf{c} \)
    \item \( \mathbf{a} \cdot (\mathbf{b} \times \mathbf{c}) = (\mathbf{a} \times \mathbf{b}) \cdot \mathbf{c} \)
    \item \( \mathbf{a} \times (\mathbf{b} \times \mathbf{c}) = (\mathbf{a} \cdot \mathbf{c})\mathbf{b} - (\mathbf{a} \cdot \mathbf{b})\mathbf{c} \)
\end{enumerate}

\paragraph{Triple Product}\mbox{}\\
\begin{equation}
    \label{triple product}
    a \cdot (b \times c) = 
    \begin{vmatrix}
        a_1 & a_2 & a_3\\
        b_1 & b_2 & b_3\\
        c_1 & c_2 & c_3
    \end{vmatrix}
\end{equation}\\
Volume of a parallelepiped: \(V = |a \cdot (b \times c)|\)\\
*Insert image here*\\

\paragraph{Equations of Lines and Planes}\mbox{}\\
\begin{doublespace}
    \(\vec{v} = \langle a,b,c \rangle\)\hspace{1cm}
    \(\vec{r} = \langle x,y,z \rangle\)\hspace{1cm}
    \(\vec{r_0} = \langle x_0,y_0,z_0 \rangle\)\\
    \(\langle x,y,z \rangle = \langle x_0 + ta, y_0 + tb, z_0 + tc \rangle\)\\
    \(x = x_0 + at\)\hspace{1cm}\(y = y_0 + at\)\hspace{1cm}\(z = z_0 + at\)\\
    Suppose that \(\vec{a}\) and \(\vec{v}\) are parallel vectors, and t is a scalar. Then:
\end{doublespace}

\begin{equation}
    \label{some identity}
    \vec{a} = t\vec{v}
\end{equation}
\begin{equation}
    \label{vector equation of L}
    \vec{r} = r_0 + t\vec{v}
\end{equation}\\
Which also means:
\begin{equation}
    \label{parametric equavalents}
    \frac{x - x_0}{a} = \frac{y - y_0}{b} = \frac{z - z_0}{c}
\end{equation}\\

Line segment from \(\vec{r_0}\) to \(\vec{r_1}\) :\\
\begin{equation}
    \mathbf{r}(t) = (1-t)\mathbf{r_0} + t\mathbf{r_1} \hspace{1cm} 0 \leq t \leq 1
\end{equation}\\

Let \(\mathbf{n}\) be a vector orthogonal (normal ie. 90 deg) to a plane P, then:\\
\begin{equation}
    \mathbf{n} \cdot (\mathbf{r - r_0}) = 0
\end{equation}\\
\begin{equation}
    \mathbf{n} \cdot \mathbf{r} = \mathbf{n} \cdot \mathbf{r_0}
\end{equation}\\

Scalar equation of the plane through point \(P_0(x_0, y_0, z_0)\) with \(\mathbf{n}\) :\\
\begin{equation}
    a(x-x_0) + b(y-y_0) + c(z-z_0) = 0
\end{equation}\\

Two planes are parallel if their normal vectors are parallel.\\

\paragraph{Cylinders and Quadric Surfaces}\mbox{}\\
*Insert images of surfaces here*\\

\begin{figure}[t]
    \centering
    \includegraphics[width=\linewidth]{table_of_surfaces}
\end{figure}


\subsection*{Chapter 13: Vector Functions}


\end{document}